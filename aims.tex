\documentclass[11pt]{article}

\usepackage[utf8]{inputenc} % Allows UTF-8 encoding
\usepackage{amsmath}        % Useful for math symbols and environments
\usepackage{geometry}       % Allows to set the page dimensions
\geometry{margin=0.5in}       % Set all margins to 1 inch

% \title{Connecting the 3D landscape of cell-free DNA (cfDNA) with its somatic mutations for the non-invasive subtyping of cancer}
\title{}
\author{}
\date{}

\begin{document}
\pagenumbering{gobble}
% \maketitle
\vspace{-1cm}

\noindent Cell-free DNA, or cfDNA, refers to fragmented DNA molecules found circulating freely in the bloodstream or other bodily fluids. 
Sequencing cfDNA molecules enables the non-invasive detection of cancer by identifying tumor-specific genetic mutations, such as through the use of GEMINI (GEnome-wide Mutational Incidence for Non-Invasive detection of cancer). \cite{bruhm_single-molecule_2023}
To better understand tumor mechanisms and subtypes, it is important to explore the relationship between somatic mutations and gene regulation in tumor versus healthy cells. Analyzing cfDNA methylation and histone modification patterns can help uncover this relationship. \cite{penny_chromatin-_2024} \\
In the proposed study, we seek to connect the 3D landscape of cfDNA with its somatic mutations by analyzing matched genetic, methylation, and histone modification data. 
Mutations are limited in that they cannot capture the tissue of origin, while epigenetic features overcome this limitation and have been shown to be stereotyped between tissues and higher in number of alterations compared to mutations. \cite{penny_chromatin-_2024} 
We specifically collect data from breast cancer patients, as breast cancer is a 
highly heterogenous disease with several subtypes that have distinct genetic and 
epigenetic profiles. Using their cfDNA genetic and epigenetic profiles, we apply 
machine learning models to predict breast cancer subtypes and infer 
regulatory mechanisms underlying these subtypes.
\section*{Determine the association of the cfDNA methylome and cancer subtypes}
Recent studies have shown that methylation patterns are highly indicative of cell type of origin.\cite{spector_methylome_2023} 
By analyzing the methylation patterns of cfDNA, we can infer the tissue of origin of the tumor and apply this towards subtyping a tumor.

\begin{enumerate}
	\item \textbf{Apply cfMethyl-seq \cite{stackpole_cost-effective_2022} and whole-genome sequencing to a breast cancer cohort to capture genome-wide methylation profiles of CpG-rich regions and mutation profiles.} We collect methylation data for both tumor and healthy samples to identify tumor-specific methylation changes. A cohort of 2000 breast cancer patients with various subtypes will be used.
	\item \textbf{Prediction of cancer subtypes based on methylation and mutation features.}
		Using the approach of GEMINI, we can calculate tumor-normal matched methylation scores over nonoverlapping genomic bins. We then will evaluate basic machine learning models in predicting cancer subtypes based on GEMINI-methylation features, GEMINI-mutation features, and a combination of both.
\end{enumerate}

\section*{Predicting tumor transcriptome through cfDNA-derived mutations and histone modification signals in promoter and enhancer regions}
Different cancer subtypes have distinct gene expression profiles. Combining cfDNA histone modification data, which is associated with gene expression programs of their cells of origin \cite{sadeh_chip-seq_2021, trier_maansson_cell-free_2023}, with mutation profiles can provide a more comprehensive view of the gene expression landscape in a tumor.
\begin{enumerate}
	\item \textbf{Collect cfCHiP-seq \cite{baca_liquid_2023} and sequence a breast cancer cohort to generate histone modification signals and mutation profiles.} We will use a cohort of 2000 patients with various cancer types and subtypes to generate H3K4me3 and H3K27ac signals. H3K4me3 and H3K27ac are histone modifications that are associated with promoters and enhancers, respectively.
	\item \textbf{Collect transcriptome data for tumor samples in the cohort.} We will collect RNA-seq data for the tumor samples in the cohort to generate gene expression profiles.
	\item \textbf{Build and evaluate machine learning models to predict gene expression profiles based on GEMINI-mutation features, H3K4me3 signals, and H3K27ac signals.} We will evaluate the performance of basic machine learning models in predicting gene expression profiles based on GEMINI-mutation features, H3K4me3 signals, and H3K27ac signals. 
\end{enumerate}

The proposed approach will provide a comprehensive view of the relationship between somatic mutations and regulatory properties of DNA, specifically methylation patterns and histone modifications. This will enable the non-invasive subtyping of cancer based on cfDNA data, providing insights into the regulatory mechanisms underlying cancer subtypes and informing the development of targeted therapies.

\newpage
\section*{Significance}
\subsection*{Paired genetic and epigenetic data for non-invasive cancer subtyping} The proposed approach will provide a comprehensive view of the relationship between somatic mutations and regulatory properties of DNA, specifically methylation patterns and histone modifications. While many studies have focused on either genetic or epigenetic cfDNA samples, there is largely a lack of studies that utilize matched genetic and epigenetic data for the same samples. Using enriched cfDNA samples, we can generate paired genetic and epigenetic data that will help understand the correspondence between genetic mutations and the 3D landscape of DNA. This will help improve our ability in several cancer-related prediction tasks, such as predicting tumor behavior, tumor growth, therapy response, and metastatic potential. The proposed approach may also motivate further work in combining other omes of cfDNA, such as fragmentomics and nucleosomics.  
\subsection*{Identification of significant regulatory regions} Regulatory regions are 
important for understanding driver mutations and gene expression programs in cancer.
Profiling cfDNA for histone modifications and methylation patterns will bring us 
closer to identifying the regulatory patterns in a tumor, as genetic mutations alone
do not provide a complete picture of the regulatory landscape. 
\subsection*{More granular subtyping for improved personalized medicine}
While two different subtypes of cancer may share the same genetic mutation profiles,
they may have distinct transcriptome and proteome profiles.
By incorporating methylation and histone modification data, we may be able to
distinguish more granular subtype differences that are not immediately apparent
in genetic mutation profiles alone. 
Identifying these differences can help in the development of more targeted therapies.
\subsection*{Advancing non-invasive liquid biopsy diagnostics} 
The proposed approach may revolutionize cancer diagnostics by enabling the
classification of a patient's subtype through a fast and non-invasive blood test.
For a heterogeneous cancers like breast cancer, this will be particularly impactful
as it will allow for more targeted and personalized treatment plans. 
Current diagnosis methods often rely on invasive tissue biopsies, 
which may not be feasible for all patients due to the risks associated with
surgery and the difficulty in obtaining tissue samples from certain parts of the body.
cfMethyl-seq and cfChIP-seq not only offer non-invasive alternatives for arriving
at similar diagnostic conclusions, but are also cost-effective and scalable.
Furthermore, identification of significant associations between methylation, mutation, histone modification events with the transcriptome of different cancer subtypes can be used to create targeted biomarker assays, which can ultimately be validated both analytically- determining the precision and sensitivity of the assay- and clinically- determining the utility in assay in informing medical decisions.
\section*{Innovation}
By combining the analyses of multiple omes of cfDNA, we are able to conduct
multiple assays on the same sample, which is a novel approach in the field of
cfDNA analysis. Our workflow in associating methylation, mutation, and histone
modification data is applied towards a heterogeneous cancer cohort, however 
this approach can be extended to other cancer types and diseases. We also
envision the potential for our machine learning models to be integrated into
clinical workflows, enabling real-time predictions of cancer subtypes. 
Future work that adopts our approach may also consider assessing epigenetic
and mutational signature changes over time, which can provide insights into
tumor evolution and monitoring of treatment response.

\newpage
\section*{Aim 1: Determine the association of the cfDNA methylome and cancer subtypes}
\subsection*{Aim 1A: Generate paired cfDNA methylation, mutation, and tumor transcriptomic profiles in a breast cancer cohort}
\subsubsection*{Rationale}
Breast cancer is a highly heterogenous disease, where epigenetic changes, such as methylation, 
play a critical role in tumor development, progression, and subtype differentiation.
The integration of cfDNA methylation profiles with mutation profiles and a transcriptomic picture of the tumor offers a comprehensive framework to elucidate the interplay between genetic and epigenetic alterations in breast cancer. 
Moreover, having matched genetic and epigenetic data allows us to both identify mutations driving tumor initiation and progression and understand how aberrant methylation patterns can regulate tumor suppressor genes and oncogenes.
\subsubsection*{Methods}
\textbf{Construct a cfDNA integrative omics dataset using a 2000-patient breast cancer cohort:}
We will analyze a cohort of 2,000 breast cancer patients representing the four major subtypes: Luminal A, Luminal B, HER2-enriched, and Triple-negative.
Blood samples will be collected from each patient, and cfDNA will be extracted using the QIAamp Circulating Nucleic Acid Kit, which provides high yield and quality for cfDNA.
Corresponding tissue biopsies will also be collected for bulk RNA sequencing to capture tumor transcriptomic data.
To link cfDNA methylation and genome sequencing data accurately, each cfDNA molecule will be tagged with a unique molecular identifier (UMI).
Tumor fraction in each cfDNA sample will be estimated using ichorCNA, and samples with low tumor fraction will be excluded from further analysis.
cfDNA concentration and quality will be validated using the Bioanalyzer 2100 to ensure suitability for downstream applications.
The extracted cfDNA will then be split for two complementary assays: cfMethyl-seq and next-generation whole-genome sequencing (WGS). cfMethyl-seq is chosen for its cost-efficiency and ability to capture genome-wide methylation profiles in CpG-rich regions, even with low DNA input. WGS will provide comprehensive genomic data, enabling a multi-omics approach to investigate genetic and epigenetic alterations across breast cancer subtypes.
\medskip \\ 
\textbf{Generation of paired GEMINI-methylation and GEMINI-mutation features:}
We apply the GEMINI method to calculated tumor-normal matched methylation and mutation
scores over nonoverlapping genomic bins.
The generation of methylation and mutation features for $J$ cancer patients and $K$ healthy patients is defined as:
\begin{equation}
	M_{i} = \frac{\sum\limits_{j \in J}{y^{T}_{ij}}}{\sum\limits_{j \in J}{x^{T}_{ij}}} - 
	\frac{\sum\limits_{k \in K}{y^{N}_{ik}}}{\sum\limits_{k \in K}{x^{N}_{ik}}},
	\quad
	G_{i} = \frac{\sum\limits_{j \in J}{z^{T}_{ij}}}{\sum\limits_{j \in J}{x^{T}_{ij}}} - 
	\frac{\sum\limits_{k \in K}{z^{N}_{ik}}}{\sum\limits_{k \in K}{x^{N}_{ik}}}
\end{equation}
where $M_{i}$ and $G_{i}$ are the methylation and mutation scores for bin $i$, respectively. $y^{T}_{ij}$ and $y^{N}_{ij}$ are the number of methylated sites in tumor and normal samples, respectively, for bin $i$. $x^{T}_{ij}$ and $x^{N}_{ij}$ are the total number of reads in tumor and normal samples, respectively, for bin $i$. $z^{T}_{ij}$ and $z^{N}_{ij}$ are the number of mutations in tumor and normal samples, respectively, for bin $i$. cfMethyl-seq only captures CpG-rich regions, thus most of the bins 
may not have any methylation signal.
Regardless, having the same binned representation for both methylation and mutation data allows for direct comparison between the two modalities for downstream analyses.

\subsection*{Aim 1B: Inferring driving gene expression programs from cfDNA-derived mutation and methylation signals}
\subsubsection*{Rationale}
With the availability of gene expression data in the tumor, we can validate the association between the methylation and mutation signals in the cfDNA and the gene expression programs in the tumor. Specifically, we can determine whether specific methylation events or genetic mutations are associated with the upregulation or downregulation of specific genes. This information can provide insights into the regulatory mechanisms underlying each breast cancer subtype.
\subsubsection*{Methods}
\textbf{Identify hypermethylated and hypomethylated activity in promoter and enhancer regions} We will isolate the methylation signals in the promoter and enhancer regions of the genome, as these regions are known to be most relevant for gene expression regulation. 
We will then identify, for each cancer subtype, differentially methylated regions (DMRs) between the tumor and normal samples using Dispersion Shrinkage for Sequencing (DSS), a package for detecting differentially methylated regions from WGBS data. 
\medskip \\
\textbf{Associate regulatory regions' methylation and mutation signals with gene expression programs:} We will correlate the methylation and mutation signals in the regulatory regions with their corresponding gene expression programs in the tumor samples. We will also validate specific DMRs using annotations from the MENT and PubMeth databases, which provide prior knowledge on the regulatory roles of specific methylation events in cancer.

\subsection*{Aim 1C: Prediction of cancer subtypes based on methylation and mutation features}
\subsubsection*{Rationale}
Following the generation of GEMINI-methylation and GEMINI-mutation features,
we are able to determine the association of the cfDNA methylome, genetic mutations,
and breast cancer subtypes. Using machine learning models, we can evaluate the 
predictive power of these features in distinguishing between the subtypes.
Furthermore, we can explore the association of methylation events with specific
genetic mutations, providing insights into the regulatory mechanisms underlying
each subtype.
\subsubsection*{Methods}
\textbf{Unsupervised clustering of methylation and mutation features:}
Let $M$ be the matrix of methylation features, of shape $n_M \times m$, where $n_M$ is the number of samples with methylation signals and $m$ is the number of bins.
Similarly, let $G$ be the matrix of mutation features, of shape $n_G \times m$, where $n_G$ is the number of samples with mutation signals. 
To visualize groupings of samples based on these profiles,
we apply UMAP to matrix $M$ and $G$ to reduce the dimensionality of the data to a 2D space.
We can then plot the UMAP embeddings and color the points by tumor vs. normal status
to see if there is a clear separation between the two groups.
After filtering out normal samples, we can then color the points by cancer subtype to see if there is a clear separation between the subtypes.
Finally, we can apply a similar analysis to the combined methylation and mutation features to see if the two modalities provide complementary information in distinguishing between the subtypes.
\medskip \\
\textbf{Supervised classification of cancer subtypes:}
We will evaluate the performance of basic machine learning models in predicting the 
breast cancer subtypes based on GEMINI-methylation features, GEMINI-mutation features, and a combination of both. 
We first will apply a simple multinomial logistic regression model to predict the subtypes based on the methylation and mutation features, which will serve as a baseline model.
The benefit of using a logistic regression model is that it is interpretable and can provide insights into the importance of each feature in predicting the subtypes.
We will then evaluate more complex models, including random forests and neural networks to see if they can improve the predictive performance.


\subsection*{Expected Results and Interpretation}
We expect to find the methylation and mutation signals to have significant differences between the tumor and normal samples. This is expected as tumor samples are known
to have hyper-methylation events and higher mutation frequencies compared to normal samples.
We also expect to see differences in the methylation and mutation signals between the different cancer subtypes.
Different subtypes of breast cancer are characterized by distinct genetic and 
epigenetic profiles, and we expect these differences to be reflected in the methylation and mutation signals.
We also expect to see a correlation between some methylation events and specific genetic mutations. Some mutations, particularly in promoter and enhancer regions, 
affect the methylation patterns of nearby CpG sites.
For the unsupervised clustering analysis, we expect to see a fairly 
clear separation between the tumor and normal samples, and a fairly 
clear separation between the different cancer subtypes. 
Because different subtypes are characterized by different 
genetic and epigenetic profiles, we expect these differences to be
reflected in the UMAP embeddings for the corresponding feature sets. 
For the supervised classification task, we hypothesize that combining
the methylation and mutation features will improve the predictive performance
because the two modalities should provide complementary information in categorizing
the subtypes.



\subsection*{Potential Pitfalls and Alternative Approaches}
For Aim 1A, One of the primary challenges in collecting matched genetic and epigenetic data is
having a sufficient yield and quality of cfDNA to perform both assays. 
Without significant enrichment of cfDNA, it may be difficult to detect methylation
signals and mutations in some samples. We chose the QIAamp Circulating Nucleic Acid Kit
because it has been shown to have one of the highest yields of cfDNA compared to other
methods. In the event that insufficient cfDNA is obtained, we will consider using
PCR-based methods, such as methylation-specific PCR, to enrich for the methylation signalprior to sequencing and the generation of GEMINI-methylation and GEMINI-mutation features. 
For Aim 1B, we may observe a lack of clear separation between the different cancer subtypes and consequently, the models may not perform well in predicting the subtypes. This could be due to overlapping biological characteristics, noise in cfDNA data, or the lack of sufficient discriminatory power in either feature set. Furthermore, the combined methylation and mutation features may introduce redundancy or noise, potentially degrading model performance instead of improving it. This could result in both unsupervised clustering and supervised classification failing to clearly differentiate subtypes.
To address this potential pitfall, we will consider applying various denoising approaches, including feature selection and batch effect correction. We can also adjust the bin size adopted for the GEMINI-methylation and GEMINI-mutation features to reduce noise and redundancy.
\newpage

\bibliographystyle{plain}
\bibliography{aims.bib}
\nocite{*}
\end{document}

